\section{Семинар 2}

Докажем регулярность второго примера из лекции

\begin{proof}
  $\underbrace{F}_{\text{замк.}} \subset A \subset \underbrace{G}_{\text{откр.}}$

  $$\mu(G) - \epsilon \leq \mu(A) \leq \mu(F) + \epsilon$$

  $\alpha: \R \to \R$ --- возрастающая

  $A = (a, b]$

  $\mu(A) = \alpha(b+) - \alpha(a+)$

  $G = (a, b+\delta)$

  $\mu(G) = \alpha((b + \delta)-) - \alpha(a+)$

  $\mu(G) - \mu(A) = \alpha((b + \delta)-) - \alpha(b+) \leq \alpha(b + \delta) - \alpha(b+) \to 0$, при $\delta \to 0$.

  Аналогично для $F$.
\end{proof}

\begin{theorem}
  Любое открытое множество можно представить в виде счетного числа открытых параллелепипедов.
\end{theorem}

\begin{proof}
  %сюда бы картиночку
  Сначала разбиваем пространство на единичные кубы. Выбираем из них все те кубы (с замыканием), лежащие внутри множества.
  Те кубы, которые пересекают наше множество, разделим на $2^n$ частей. Берем среди кубов поменьше выберем все лежащие внутри.
  Повторяем это процесс до бесконечности. Утверждается, что объединение всех таких кубиков --- открытое множество. 
  Для любой точки множества достаточно рассмотреть кубик с ребром меньшим, чем расстояние до границы. 
\end{proof}

\begin{problem}
  $$d(A, B) = \mu^*(A \triangle B)$$
  $$d(A, B) \leq d(A, C) + d(C, B)$$
\end{problem}

\begin{proof}
  Докажем утверждение через монотонность меры.
  Достаточно просто доказать, что 

  $$A\triangle B \subset (A\triangle C) \cup (C\triangle B)$$

  Если $x \in A, x\notin B$, то $x \in (A\backslash C) \cup (C \backslash B) \subset (A\triangle C) \cup (C\triangle B)$
  
  Если $x \in B, x\notin A$ аналогично.
\end{proof}


\subsection{Неравенства: продолжения}

\begin{theorem}[Минковский]
  $$\left(\sum_{i = 1}^{n} |a_i + b_i|^p\right)^\frac{1}{p} = \left(\sum_{i = 1}^{n} |a_i|^p\right)^\frac{1}{p} + \left(\sum_{i = 1}^{n} |b_i|^p\right)^\frac{1}{p}$$
\end{theorem}

Для самостоятельного вывода

