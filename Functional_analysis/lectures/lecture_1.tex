\newpage

\section{Элементы теории меры и интеграла}

\textbf{Мера} --- счетно-аддитивная функция множества.

\begin{definition}
  $\mathcal{R}$ --- семейство множеств называется \textit{кольцом}, если $A,\, B \in \mathcal{R} \Rightarrow A \cup B,\ A \backslash B \in \mathcal{R}$
\end{definition}

\begin{definition}
  Кольцо называют $\sigma$-кольцом, если дополнительно $\sum_{n = 1}^{\infty} A_n \in \mathcal{R}$ при $A_n \in \mathcal{R}$. 
\end{definition}

\begin{definition}
  Функция $\phi : \mathcal{R} \to [0, +\infty]$ называется \textit{аддитивной}, если $\forall A, B \in \mathcal{R} : A\cap B = \emptyset$ выполнено
  $\phi(A\cup B) = \phi(A) + \phi(B)$.
\end{definition}

Некоторые очевидные свойства

\begin{itemize}
  \item $\phi(\emptyset) = 0$
  \item $\phi(\bigcup_{n = 1}^{\infty} A_n) = \sum_{n = 1}^{\infty} \phi(A_n)$,
        при $A_i \cap A_j = \emptyset (i \ne j)$ и $A_i \in \mathcal{R}$
  \item $\phi(A\cup B) + \phi(A \cap B) = \phi(A) + \phi(B)$
  \item $A \subset B \Rightarrow \phi(A) \leq \phi(B)$
  \item $A \subset B$ и $\phi(A) < +\infty$ $\Rightarrow$ $\phi(B\backslash A) = \phi(B) - \phi(A)$
\end{itemize}

\begin{definition}
  $\mathcal{R}$ --- $\sigma$-кольцо, $\phi: \mathcal{R} \to [0, +\infty]$

  $\phi$ называется \textit{счётно-аддитивным}, если

  $A_n \in \mathcal{R}, A_i \cap A_j = \emptyset (i \ne j) \Rightarrow \phi(\bigcup_{n=1}^{\infty} A_n) = \sum_{n = 1}^{\infty} \phi(A_n)$
\end{definition}

Счетно-аддитивная функция удовлетворяет свойству непрерывности в следуюющем смысле:

\begin{itemize}
  \item $A_1 \subset A_2 \subset \dots \subset A_n \dots$, $A = \bigcup A_n \Rightarrow \phi(A_n)\to \phi(A)$, при $n \to \infty$
\end{itemize}

$$B_1 = A_1,\ B_2 = A_2 \backslash A_1,\ B_3 = A_3 \backslash A_2,\ \dots$$

$$B_i \cap B_j = \emptyset (i \ne j),\ A_n = \bigcup_{i = 1}^{n} B_n,\ A = \bigcup_{i=1}^{\infty} B_n$$

$$\phi(A_n) = \phi(B_1) + \dots \phi(B_n) \Rightarrow \phi(A) = \sum_{n = 1}^{\infty} \phi(B_n)$$


\begin{definition}
  Мера --- неотрицательная, счетно-аддитивная функция, определенная на $\sigma$-алгебре.
\end{definition}




\subsection{Мера Лебега в евклидовом пространстве $\R^n$}

\begin{definition}
  Брус --- $I = \left\{x \in \R^n : a_i \leq x_i \leq b_i \right\}$.
\end{definition}

Примеры:

1) $\mu(I) = \prod_{i = 1}^{n} (b_i - a_i)$

2) $p = 1$, $\R^n = \R$, $\alpha : \R \to \R$ --- монотонно-возраст

\begin{itemize}
  \item $\mu([a, b]) = \alpha(b+) - \alpha(a-)$
  \item $\mu((a, b)) = \alpha(b-) - \alpha(a+)$
  \item $\mu([a, b)) = \alpha(b-) - \alpha(a-)$
  \item $\mu((a, b]) = \alpha(b+) - \alpha(a+)$
\end{itemize}

\begin{definition}
  \textit{Элементарное множество} --- конечное объединение брусов. Обозначаем $A \in \mathcal{E}$
\end{definition}

Люблое элементарное множество можно представить в виде конечного объединение непересекающихся брусов.


$\mu(A)$, $A \in \mathcal{E}$, $\mu(A) = \sum \mu(I_i)$ при $I_i \cap I_j = \emptyset (i \ne j)$

Результат $\mu(A)$ не зависит от разбиения $A$. А функция $\mu$ оказывается аддитивной.

$\mathcal{E}$ является кольцом, а $\mu$ аддитивна на нем.

\begin{problem}
  Любое открытое множество в $\R^n$ можно представить в виде счетного числа брусов.
\end{problem}

$0 \leq \mu$, конечность, аддитивность и регулярность на $\mathcal{E}$.

\begin{definition}
  $\mu$ называется регулярной (на $\mathcal{E}$), если 

  $\forall A \in \mathcal{E}$ и $\forall \epsilon>0\ \exists\text{ замкнутые } F \in \mathcal{E} \text{ и открытое } G \in \mathcal{E}$

  $F \subset A \subset G$ и $\mu(G) - \epsilon \leq \mu(A) \leq \mu(F) + \epsilon$
\end{definition}