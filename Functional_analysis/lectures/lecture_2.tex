\begin{definition}
  \textit{Внешняя мера} произвольного множества $E \in \R^n$ $\mu^*(E)$ определяется как:
  $$\mu^*(E) = \inf \sum_{i = 1}^{\infty} \mu(A_i),$$
  где инфинум берется по всевозможным, не более чем счетными покрытиям
  $\{A_i\},\; A_i \in \mathcal{E}$ множества $E$.
\end{definition}

\textbf{Замечание.} В определении можно заменить открытые элементарные множества на открытые параллелепипеды.

\begin{lemma}
  Внешняя мера является монотонной:
  $$E_1 \subset E_2 \Rightarrow \mu^*(E_1) \leq \mu^*(E_2)$$
\end{lemma}

2 свойства внешней меры
\begin{itemize}
  \item $\mu^*(A) = \mu(A)$, для $A \in \mathcal{E}$
  \item $\mu^*\left( \bigcup_{n = 1}^{\infty} E_n \right) \leq \sum_{n=1}^{\infty} \mu^*(E_n)$
\end{itemize}

\begin{proof}
  \phantom{123}\par
  \begin{itemize}
  \item[1)] \textbf{$\leq$} 
  
  $A \in \mathcal{E}$, $\epsilon > 0$, берем открытое $G \in \mathcal{E}$: $A\subset G$ и 
  $\mu(G) \leq \mu(A) + \epsilon$ (регулярность меры).

  По определению внешней меры $\mu^*(A) \leq \mu(G) \leq \mu(A) + \epsilon$.
  
  В силу произвольности $\epsilon$ $\Rightarrow$ $\mu^*(A) \leq \mu(A)$.

  \vspace{2mm}
  \textbf{$\geq$} 
  
  $\mu^*(A) \leq \infty \; \Rightarrow\; \forall \epsilon > 0$ находится покрытие, что 
  $$\sum_{n=1}^{\infty} \mu(A_n) \leq \mu^*(A) + \epsilon$$

  Находим замкнутое элементарное $F \subset A$, $\mu(A) \leq \mu(F) + \epsilon$.
  При этом $F$ --- компакт. Значит, $\exists N: \; F \subset A_1 \cup \ldots \cup A_N$.
  $$\mu(A) \leq \mu(F) + \epsilon \leq \mu(A_1 \cup \ldots \cup A_N) + \epsilon \leq \sum_{n = 1}^{\infty} \mu(A_n) + \epsilon \leq \mu^*(A) + 2\epsilon$$

  В силу произвольности $\epsilon$ получаем утверждение.

  \vspace{2mm}
  \item[2)] $$\mu^*\left(\bigcup_{n = 1}^{\infty} E_n\right) \leq \sum_{n = 1}^{\infty} \mu^*(E_n)$$
  
  Если мера хоть какого-то $\infty$, то доказывать нечего.

  $\forall n \; \mu^*(E_n) < \infty$

  $\epsilon>0$, $E_n \subset \bigcup_{k = 1}^{k_n} A_{nk}$

  $$\sum_{k = 1}^{k_n} \mu(A_{nk}) \leq \mu^*(E_n) + \frac{\epsilon}{2^n}$$

  $$\bigcup_n E_n \subset \bigcup_{n, k} A_{nk}$$

  $$\mu^*\left(\bigcup_{n = 1}^{\infty} E_n\right) \leq \sum_{n = 1}^{\infty} \sum_{k = 1}^{k_n} \mu(A_{nk}) \leq \sum_{n=1}^{\infty}\left( \mu^*(E_n) + \frac{\epsilon}{2^n}\right) = \sum_{n = 1}^{\infty} \mu^*(E_n) + \epsilon$$

  В силу произвольности $\epsilon$ получаем искомое.
  \end{itemize}
\end{proof}

\begin{definition}
  $d(A, B) = \mu^*\big((A\backslash B) \cup (B \backslash A)\big)$
\end{definition}

$d$ удовлетворяет некоторым аксиомам метрики

\begin{itemize}
  \item $d(A, B) = d(B, A)$
  \item $d(A, A) = 0$
  \item $d(A, B) \leq d(A, C) + d(C, B)$
  \item $d(A, B) = 0$ \textbf{не} означает, что $A \equiv B$
\end{itemize}

Некоторые свойства

\begin{itemize}
  \item $\mu^*(A) = d(A, \emptyset)$
  \item $|\mu^*(A) - \mu^*(B)| \leq d(A, B)$
  \item $d(A_1 \cup A_2, B_1 \cup B_2) \leq d(A_1, B_1) + d(A_2, B_2)$
  \item $d(A_1 \cap A_2, B_1 \cap B_2) \leq d(A_1, B_1) + d(A_2, B_2)$
  \item $d(A_1 \backslash A_2, B_1 \backslash B_2) \leq d(A_1, B_1) + d(A_2, B_2)$
\end{itemize}

\begin{definition}
  Множество $A \in \R^p$ называется \textit{конечно измеримым}, если $$\exists \{A_n \in \mathcal{E}\}_{n = 1}^{\infty}: d(A, A_n) \to 0, \text{ при } n \to \infty.$$
\end{definition}
\begin{definition}
  Подмножество $\R^p$ называется измеримым, если оно представимо в виде не более чем счетного объединения конечно измеримых множеств.
\end{definition}

$\mathbb{M}_F(\mu)$ --- всевозможные объединения конечно измеримых множеств

$\mathbb{M}(\mu)$ --- все измеримые множества

\begin{lemma}
  \hypertarget{Lemma1}{}
  $$A \in \mathbb{M}(\mu), \mu^*(A) < \infty  \Rightarrow A \in \mathbb{M}_F(\mu)$$
\end{lemma}

\begin{theorem}[Каратеодори]
  $\mathbb{M}(\mu)$ --- $\sigma$-алгебра, а $\mu^*$ на ней счетно-аддитивна.
\end{theorem}

\begin{proof}
  \phantom{}\par
  \begin{itemize}
  \item[1)] Докажем, что $\mathbb{M}_F(\mu)$ --- кольцо, а $\mu^*$ аддитивна на $\mathbb{M}_F(\mu)$.

  Возьмем $A, B \in \mathbb{M}_F(\mu)$.

  $d(A, A_n), d(B, B_n) \to 0$ при $ n\to \infty$ (аппроксимируем элементарными множествами)

  $\mu^*(A_n) \to \mu^*(A)$, $\mu^*(B_n) \to \mu^*(B)$

  Из свойств функции $d$:
  $$A_n \cup B_n \to A\cup B \quad A_n \cap B_n \to A\cap B \quad A_n \backslash B_n \to A\backslash B$$

  $$\mu(A_n \cap B_n) + \mu(A_n \cup B_n) = \mu(A_n) + \mu(B_n)$$
  $$\mu^*(A_n \cap B_n) + \mu^*(A_n \cup B_n) = \mu^*(A_n) + \mu^*(B_n)$$

  Предельным переходом получим

  $$\mu^*(A \cap B) + \mu^*(A \cup B) = \mu^*(A) + \mu^*(B)$$

  \item[2)] $A = \bigcup_{n =1}^{\infty} A_n$, $A_n \in \mathbb{M}_F(\mu)$, причем $A_i$ не пересекаются.

  $$\mu^*(A) \leq \sum_{n = 1}^{\infty} \mu^*(A_n)$$

  $B_n = A_1 \cup \ldots \cup A_n \subset A$

  $\mu^*(B_n) = \mu^*(A_1) + \mu^*(A_2) + \ldots + \mu^*(A_n) \leq \mu^*(A)$

  $n \to \infty: \sum_{n=1}^{\infty} \mu^*(A_n) \leq \mu^*(A)$

  Осталось распространить равенство на случай, когда $A_n$ просто измеримы (не обязательно конечно). 
  Для этого достаточно воспользоваться \hyperlink{Lemma1}{Леммой 1.2.}
  \end{itemize}
\end{proof}