\subsection{Дифференцируемость решений по начальным данным и параметрам}

$$\frac{dx}{dt} = f(t, x, \mu), \quad \mu = \begin{pmatrix}
  \mu_1 \\
  \vdots \\
  \mu_m
\end{pmatrix} \in \R^m$$

$x(t, \tau, \xi, \mu)$ --- решение задачи Коши с начальными данными $(\tau, \xi)$.

\begin{theorem}
  Пусть $f\in C_1(G \times\mathfrak{M})$. Тогда решение задачи Коши $x(t,\tau, \xi, \mu)$
  с начальными данными $(\tau, \xi)$ непрерывно дифференцируемо по всем своим аргументам.
\end{theorem}

\begin{proof}
  Будем использовать, что:
  \begin{itemize}
    \item $x(t, \tau, \xi, \mu)$ решение \\ $$\frac{d}{dt}x(t, \tau, \xi, \mu)
    = f(t, x(t, \tau, \xi, \mu), \mu);$$
    \item $x(t, \tau, \xi, \mu)$ решение задачи Коши с начальными данными
    \\$$x(\tau, \tau, \xi, \mu) = \xi.$$
  \end{itemize}

  По факту, хотим доказать, что:

  $$\frac{\partial x}{\partial t},\quad \frac{\partial x}{\partial \tau},
  \quad \frac{\partial x}{\partial \xi},\quad \frac{\partial x}{\partial \mu}$$
  
  существуют и непрерывны. Первое уже не раз доказывалось.

  1) $\dfrac{\partial x}{\partial \tau} = v(t)$ --- функция в $\R^n$

  $$\left. \frac{\partial}{\partial \tau}\quad \right|\quad \frac{dx}{dt} = f(t, x(t, \tau, \xi, \mu))$$
  $$\frac{\partial}{\partial \tau} \frac{dx}{dt} =
  \frac{\partial f}{\partial \tau} =
  \frac{\partial f}{\partial t} \underbrace{\frac{\partial t}{\partial \tau}}_{= 0}+
  \frac{\partial f}{\partial x} \underbrace{\frac{\partial x}{\partial \tau}}_{= v} +
  \frac{\partial f}{\partial \mu} \underbrace{\frac{\partial \mu}{\partial \tau}}_{= 0}$$

  $$\frac{\partial}{\partial \tau} \frac{dx}{dt} = \frac{\partial f}{\partial x}\cdot v$$

  Т.к. все принадлежит $C_1$, то 1 и 2 множители можно поменять местами.

  \begin{equation}
    \label{eq:1}
    \frac{d}{dx} \frac{\partial x}{\partial \tau} = \frac{dv}{dt}
    = \frac{\partial f}{\partial \tau}\cdot v
  \end{equation}

  $$\frac{\partial f}{\partial x} = 
  \begin{pmatrix}
    \frac{\partial f_1}{\partial x_1} & \dots  & \frac{\partial f_1}{\partial x_n} \\
    \vdots                            & \vdots & \vdots                            \\
    \frac{\partial f_n}{\partial x_1} & \dots  & \frac{\partial f_n}{\partial x_n} \\
  \end{pmatrix}
  $$

  Вспомним, что $x$ еще и решение задачи Коши.

  $$ \left. \frac{\partial}{\partial \tau}\quad \right| \quad
  \left.x(t, \tau, \xi, \mu) \right|_{t = \tau} = \xi$$
  $$\left.\frac{\partial x}{\partial t} \frac{\partial t}{\partial \tau} \right|_{t = \tau} +
  \frac{\partial x}{\partial \tau} \frac{\partial \tau}{\partial \tau} +
  \frac{\partial x}{\partial \xi} \underbrace{\frac{\partial \xi}{\partial \tau}}_{=0} +
  \frac{\partial x}{\partial \mu} \underbrace{\frac{\partial \mu}{\partial \tau}}_{= 0} =
  \underbrace{\frac{\partial \xi}{\partial \tau}}_{=0}$$

  $$\frac{\partial x}{\partial t} + \frac{\partial x}{\partial \tau} = 0$$
  
  \begin{equation}
    \label{eq:2}
    f(\tau, \xi, \mu) = -v(\tau)
  \end{equation}

  Объединяя \ref*{eq:1} и \ref*{eq:2}, получаем задачу Коши для линейной системы.

  $$
    \begin{cases}
      \frac{dv}{dt} = \frac{\partial f}{\partial x} \cdot v \\
      v(\tau) = -f(\tau, \xi, \mu)
    \end{cases}
  $$

  Значит, можно найти $v = \dfrac{\partial x}{\partial \tau}(t, \tau, \xi, \mu)$.

  \begin{example}
    $$
    \begin{cases}
      \frac{dx}{dt} = \mu \cdot x \\
      x(\tau, \tau, \xi, \mu) = \xi
    \end{cases}
    $$
  
    $$x(t, \tau, \xi, \mu) = \xi e^{\mu(t - \tau)}$$
  
    $$\frac{\partial x}{\partial \tau} = -\mu \xi e^{\mu(t - \tau)}$$
  
    С другой стороны
    \begin{itemize}
      \item $f(t, \tau, \xi, \mu) = \mu \cdot x \Rightarrow \frac{\partial f}{\partial x} = \mu$
      \item $v(\tau) = -f(\tau, \xi, \mu) = -\mu \xi$
    \end{itemize}
  
    Получаем то же самое решение
  \end{example}

  2) Теперь считаем по $\xi$

  $w(t) = \dfrac{\partial x}{\partial \xi}$ --- Якобиан ($\M_{n \times n}$)

  $$\left. \frac{\partial}{\partial \xi}\quad \right| \quad \frac{dx}{dt}(t, \tau, \xi, \mu) = f(t, x, \mu)$$

  $$\frac{\partial}{\partial \xi} \frac{dx}{dt} =
  \frac{\partial f}{\partial t} \underbrace{\frac{\partial t}{\partial \xi}}_{= 0} +
  \frac{\partial f}{\partial x} \underbrace{\frac{\partial x}{\partial \xi}}_{= w} +
  frac{\partial f}{\partial \mu} \underbrace{\frac{\partial \mu}{\partial \xi}}_{=0}$$
  
  Первое и третье слагаемые 0 в силу независимости от $\xi$. Также меняем $\frac{d}{dt}$ и $\frac{\partial}{\partial \xi}$ местами в силу $C_1$.
  
  $$\frac{dw}{dt} = \frac{\partial f}{\partial x}w$$
  
  Снова пользуемся, что $x$ --- решение.
  $$\left. \frac{\partial}{\partial \xi}\quad \right| \quad x(t, \tau, \xi,\mu) |_{t = \tau} = \xi$$
  
  $$\left.\frac{\partial x}{\partial t} \underbrace{\frac{\partial t}{\partial \xi}}_{=0} \right|_{t = \tau} +
  \frac{\partial x}{\partial \tau} \underbrace{\frac{\partial \tau}{\partial \xi}}_{=0}  +
  \underbrace{\frac{\partial x}{\partial \xi}}_{= w} \frac{\partial \xi}{\partial \xi} +
  \frac{\partial x}{\partial \mu} \frac{\partial \mu}{\partial \xi}  =
  \frac{\partial \xi}{\partial \xi} \cdot (E_n)$$
  
  В силу независимости $\tau$, $t$ и $\mu$ от $\xi$ получим, что

  $$w(t) = E_n$$

  Снова получаем задачу Коши относительно $w$

  $$
  \begin{cases}
    \frac{dw}{dt} = \frac{\partial f}{\partial x} \cdot w \\
    w(\tau) = E_n
  \end{cases}
  $$

  \begin{example}
    $$
    \begin{cases}
      \frac{dx}{dt} = \mu \cdot x \\
      x(\tau, \tau, \xi, \mu) = \xi
    \end{cases}
    $$
  
    $$x(t, \tau, \xi, \mu) = \xi e^{\mu(t - \tau)}$$
  
    $$\frac{\partial x}{\partial \xi} = e^{\mu(t - \tau)}$$
  
    С другой стороны
    \begin{itemize}
      \item $f(t, \tau, \xi, \mu) = \mu \cdot x \Rightarrow \frac{\partial f}{\partial x} = \mu$
      \item $w(\tau) = E_1 = 1$
    \end{itemize}
  
    Получаем то же самое решение
  \end{example}


  3) $z(t) =\dfrac{\partial x}{\partial \mu}$ --- Якобиан ($\M_{n\times m}$)

  $$\left. \frac{\partial}{\partial \mu}\quad \right| \quad \frac{dx}{dt}(t, \tau, \xi, \mu) = f(t, x,mu)$$

  $$\frac{\partial}{\partial \mu} \frac{dx}{dt} =
  \frac{\partial f}{\partial t} \underbrace{\frac{\partial t}{\partial \mu}}_{=0} +
  \frac{\partial f}{\partial x} \underbrace{\frac{\partial x}{\partial \mu}}_{=z} +
  \frac{\partial f}{\partial \mu} \frac{\partial \mu}{\partial \mu}$$

  Снова меняем местами дифференциалы в левой части.
  $$\frac{dz}{dt} = \frac{\partial f}{\partial x}\cdot z + \frac{\partial f}{\partial \mu}$$

  Начальные условия

  $$\left. \frac{\partial}{\partial \mu}\quad \right| \quad
  x(t, \tau, \xi,\mu) \big|_{t = \tau} = \xi$$
  
  $$\left.\frac{\partial x}{\partial t} \underbrace{\frac{\partial t}{\partial \mu}}_{=0} \right|_{t = \tau} +
  \frac{\partial x}{\partial \tau} \underbrace{\frac{\partial \tau}{\partial \mu}}_{=0} +
  \frac{\partial x}{\partial \xi} \underbrace{\frac{\partial \xi}{\partial \mu}}_{= 0} +
  \underbrace{\frac{\partial x}{\partial \mu}}_{=z} \frac{\partial \mu}{\partial \mu} =
  \frac{\partial \xi}{\partial \mu}$$

  Первое, второе и третье слагаемые 0.

  $z(\tau) = 0_{(n \times m)}$

  Итого мы и в третий раз получили задачу Коши

  $$
  \begin{cases}
    \frac{dz}{dt} = \frac{\partial f}{\partial x} \cdot z + \frac{\partial f}{\partial \mu} \\
    z(\tau) = 0_{n \times m}
  \end{cases}
  $$

  \begin{example}
    $$
    \begin{cases}
      \frac{dx}{dt} = \mu \cdot x \\
      x(\tau, \tau, \xi, \mu) = \xi
    \end{cases}
    $$
    
    Для самостоятельного решения
  \end{example}

\end{proof}

