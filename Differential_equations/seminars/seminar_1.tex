\newpage

\section{Семинар 1}

\def \ycom {y_{\text{о.о.}}}
\def \ysp {y_{\text{ч.}}}

\subsection{Уравнение Эйлера}

\textbf{Дифференциальное уравнение Эйлера} имеет вид

$$a_n x^n y^{(n)} + a_{n-1} x^{n-1} y^{(n-1)} + \dots + a_1 x y' + a_0 y = f(x)$$

Так же, как и обычно, решения ищутся в виде $y = \ycom + \ysp$, где $\ycom$ --- общее решение однородного дифф. уравнения

$$a_n x^n y^{(n)} + a_{n-1} x^{n-1} y^{(n-1)} + \dots + a_1 x y' + a_0 y = 0,$$

а $\ysp$ --- любое частное решение исходного уравнения.

Существует 2 пути решения уравнения Эйлера:

\begin{itemize}
  \item Через замену $|x| = e^t$
  \item Через замену $y = x^\lambda$
\end{itemize}

Будем рассматривать решение уравнения 2 порядка

1) $x = e^t$

$y(t) = y(x(t))$

$y'(t) = y'(x) \cdot x'(t) = y'x$

$y''(t) = (y'x)'(t) = y''x^2 + y'x$

Тогда наше уравнение принимает вид

$$ a_2 y''(t) + (a_1 - 1)y'(t) + a_0 y(t) = 0$$

А такое уравнение мы уже умеем решать

2) $y = x^k$

$y' = kx^{k-1}$

$y'' = k(k-1)x^{k-2}$

Тогда наше уравнение принимает вид

$$ a_2 x^k k(k-1) + a_1x^k k + a_0 = 0$$

$$a_2k(k-1) + a_1k + a_0 = 0$$

$k$ находить мы умеем.

\textbf{Замечание.} Находить решения во первом случае приятнее, поскольку 
\begin{itemize}
  \item При методе вариации постоянных мы берем производные от более приятных функций
  \item После приведения характирестического многочлена относительно $x(t)$ легко находится характеристический многочлен относительно $t$,\\
  что тоже помогает при методе вариации постоянных.
\end{itemize}

Примеры задач

\begin{problem}
  $$xy'' + 2\frac{y}{x} = ln(-x)$$

  0) Приведем уравнение к "удобоворимому" виду

  $$x^2y'' + 2y = xln(-x)$$

  1) Рассматриваем характеристический многочлен и находим общее решение однородного
  
  $$\lambda(\lambda - 1) + 2 = 0$$

  $$\lambda_1 = -1,\quad \lambda_2 = 2$$

  $$\ycom(t) = C_1e^{-t} + C_2e^{2t}$$

  2) Вспоминаем, что $-x = e^t$.
    Обращаемся к характеристическому многочлену для $y(t)$.
    Построим от него неоднородный диффур.

  $$\lambda^2 - \lambda + 2 = 0$$

  $$y''(t) - y(t) + 2 = x(t)\,ln(-x(t)) = -te^t$$
\end{problem}